% -*- coding: utf-8; -*-
% vim: set fileencoding=utf-8 :
\documentclass[english,submission]{programming}
%% First parameter: the language is 'english'.
%% Second parameter: use 'submission' for initial submission, remove it for camera-ready (see 5.1)

\usepackage[backend=biber]{biblatex}
\addbibresource{example.bib}

\begin{document}

\title{(Functional) Programming Strategies}
\subtitle{Systematic Software Engineering}% optional
%\titlerunning{Preparing Articles for Programming} %optional, in case that the title is too long; the running title should fit into the top page column

\author[a,b]{Noel Welsh}
\authorinfo{is a consultant helping companies succeed with functional programming. Contact him at
  \email{noel@noelwelsh.com}.}
\affiliation[a]{Inner Product LLC, USA}
\affiliation[b]{Underscore Consulting LLP, UK}
% \authorrunning{T. Pape, C. Lopes, R. Hirschfeld} % Optional, for long author lists

\keywords{pedagogy, functional programming, CS education} % please provide 1--5 keywords


%%%%%%%%%%%%%%%%%%
%% These data MUST be filled for your submission. (see 5.3)
\paperdetails{
  %% perspective options are: art, sciencetheoretical, scienceempirical, engineering.
  %% Choose exactly the one that best describes this work. (see 2.1)
  perspective=art,
  %% State one or more areas, separated by a comma. (see 2.2)
  %% Please see list of areas in http://programming-journal.org/cfp/
  %% The list is open-ended, so use other areas if yours is/are not listed.
  area={Programming education},
  %% You may choose the license for your paper (see 3.)
  %% License options include: cc-by (default), cc-by-nc
  % license=cc-by,
}
%%%%%%%%%%%%%%%%%%

%%%%%%%%%%%%%%%%%%
%% These data are provided by the editors. May be left out on submission.
%\paperdetails{
%  submitted=2016-08-10,
%  published=2016-10-11,
%  year=2016,
%  volume=1,
%  issue=1,
%  articlenumber=1,
%}
%%%%%%%%%%%%%%%%%%

%%%%%%%%%%%%%%%%%%%%%%%%%%%%%
% Please go to https://dl.acm.org/ccs/ccs.cfm and generate your Classification
% System [view CCS TeX Code] stanz and copy _all of it_ to this place.
%% From HERE
\begin{CCSXML}
<ccs2012>
<concept>
<concept_id>10002944.10011122.10003459</concept_id>
<concept_desc>General and reference~Computing standards, RFCs and guidelines</concept_desc>
<concept_significance>300</concept_significance>
</concept>
<concept>
<concept_id>10010405.10010476.10010477</concept_id>
<concept_desc>Applied computing~Publishing</concept_desc>
<concept_significance>300</concept_significance>
</concept>
</ccs2012>
\end{CCSXML}

\ccsdesc[300]{General and reference~Computing standards, RFCs and guidelines}
\ccsdesc[500]{Applied computing~Publishing}

% To HERE
%%%%%%%%%%%%%%%%%%%%%%%

\maketitle

% Please always include the abstract.
% The abstract MUST be written according to the directives stated in
% http://programming-journal.org/submission/
% Failure to adhere to the abstract directives may result in the paper
% being returned to the authors.
\begin{abstract}
\end{abstract}

\section{Introduction}

\section{Programming Strategies}
\label{sec:programming-strategies}

In this section we describe in detail our programming strategies. We have nine strategies, divided into three groups:

\begin{enumerate}
\item \emph{Denotational strategies}: algebraic data types, and structural recursion.
\item \emph{Architectural strategies}: interpreters and algebras; reification; Church encoding; type classes.
\item \emph{General strategies}: read the literature; follow the types.
\end{enumerate}

For each strategy we give the conditions when it should be used; a detailed description of the strategy; and an example of its use. For some strategies we give additional notes that give context not used to the idiosyncracies of Scala.


\subsection{Denotational Strategies}

Denotational strategies are named in reference to denotational semantics\textbf{CITE}. In denotational semantics we give meaning to a program term by relating it to a mathematical concept we assume we already understand. In our denotational strategies we do the reverse: we start with a mathematical concept we want to realise and from that derive code. The majority of our students do not have a strong mathematical background so we do not emphasise this connection when teaching the strategies. In the description below we present the strategies in the same way we use with students.

\subsubsection{Algebraic Data Types}
\paragraph{Condition} Whenever data can be described using \emph{logical ors} or \emph{logical ands} it should be defined using an algebraic data type.

\paragraph{Description} Algebraic data types are our main way of defining data. If the data description meets the conditions for this strategy the code that defines the data follows immediately.

Algebraic data types consist of two sub-strategies, that for logical ors, also called \emph{sum types}, and that for logical ands, also called \emph{product types}.

For a sum type, if \texttt{A} is a \texttt{B} \emph{or} \texttt{C} then we can immediately write the corresponding Scala

\begin{lstlisting}[language=scala]
  sealed trait A
  final case class B() extends A
  final case class C() extends A
\end{lstlisting}

For a product type, if \texttt{A} is a \texttt{B} \emph{and} \texttt{C} then we can immediately write the corresponding Scala

\begin{lstlisting}[language=scala]
  final case class A(b: B, c: C)
\end{lstlisting}

\paragraph{Example}
\textbf{Example here}

\paragraph{Notes} Scala does not support algebraic data types directly, unlike many other functional programming languages. Instead they are constructed using \texttt{sealed traits} and \texttt{final case classes} as show in the description and examples. There are some subtleties in encoding algebraic data types in Scala. For example we might use a \texttt{final case object} instead of a \texttt{final case class} if our type holds no data. We do not got into these here as they are not of interest to non-Scala programmers. Direct syntax for algebraic data types is slated for the next of version of Scala, known as Dotty, which will do away with these issues\textbf{CITE}.

In mathematics we usually do not name fields in a product type. This is also the case in languages such as Haskell and O'Caml. In Scala we must always name the fields of a type.


\subsubsection{Structural Recursion}
\paragraph{Condition} Whenever implementing a transformation on an algebraic data type we can use structural recursion.

\paragraph{Description} Structural recursion gives us a skeleton for implementing any transformation on an algebraic data type. Like algebraic data types the structural recursion pattern has sub-patterns for sum and product types.

For a sum type, if \texttt{A} is a \texttt{B} \emph{or} \texttt{C} then we can immediately write the corresponding Scala

\begin{lstlisting}[language=scala]
  anA match {
    case B() => ???
    case C() => ???
  }
\end{lstlisting}

The \texttt{???} indicates code that we must complete with a problem-specific implementation.

For a product type, if \texttt{A} is a \texttt{B} \emph{and} \texttt{C} then we can immediately write the corresponding Scala

\begin{lstlisting}[language=scala]
  anA match {
    case A(b, c) => ???
  }
\end{lstlisting}


\paragraph{Example}

\paragraph{Notes} In Scala we can implement structural recursion using pattern matching or polymorphism. We have only illustrated the pattern matching approach here as it's more common in other functional programming languages.



\subsection{Architectural Strategies}

\subsubsection{Interpreters and Algebras}
\paragraph{Condition}
\paragraph{Description}
\paragraph{Example}

\subsubsection{Reification}
\paragraph{Condition}
\paragraph{Description}
\paragraph{Example}

\subsubsection{Church Encoding}
\paragraph{Condition}
\paragraph{Description}
\paragraph{Example}

\subsubsection{Type Classes}
\paragraph{Condition}
\paragraph{Description}
\paragraph{Example}


\subsection{General Strategies}

\subsubsection{Read the Literature}
\paragraph{Condition}
\paragraph{Description}
\paragraph{Example}

\subsubsection{Follow the Types}
\paragraph{Condition}
\paragraph{Description}
\paragraph{Example}

\section{Case Study: Streaming Data Processing}
\label{sec:case-study}

\section{Pedagogy}

\section{Related Work}

\section{Discussion and Conclusions}


\end{document}
